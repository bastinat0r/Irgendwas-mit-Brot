\documentclass{scrbook}

\title{Irgendwas mit Brot}

\author{bastinat0r \\ Wolfgang Keller \\ Andreas Pfohl}
% hier darf sich gern jeder eintragen, der mitgeschrieben hat.

\usepackage[utf8]{inputenc}
\usepackage[ngerman]{babel}

\begin{document}
\maketitle

\chapter*{Prolog}

Am Anfang war das Brot und das Brot war im Ofen. Es war gutes Brot. Und die Leute riefen "`Brecht das Brot und verteilt es unter Armen!"'.

Sie sollten mit ihrer Aussage Recht behalten.

\chapter{Brot an sich}

Sollte man ein Buch wirklich mit einem Metawitz beginnen? Ich\footnote{gemeint ist hiermit bastinat0r, der Gründer dieses hochgradig elitären Buchprojektes} denke nicht. Und dennoch – irgendwie hätte das schon Stil.

Auf der anderen Seite: Metawitze sind ein probates Mittel, den Leser dazu zu zwingen, den Denkapparat\footnote{vulgo auch als \emph{Gehirn} bezeichnet} einzuschalten.

Daher sollte die Frage eher lauten: ist es das Wesen eines Metawitzes wert, ihn mit einer seine Benutzung rechtfertigenden Bemerkung einzuleiten?

Einige Intelligente Menschen lieben es in der Tat, einen Sinn hinter Fragen, welche die Benutzung einer Bemerkung, die die Benutzung eines Metawitzes zur Einleitung debattieren, hinterfragen, zu vermuten.

"`We must go deeper"' -- hierfür sei dies als Werkzeug mitgegeben:
\begin{displaymath}
\lambda f.(\lambda x.f (x x)) (\lambda x.f (x x)).
\end{displaymath}
% Nein, das wollen wir mal lieber per Default deaktiviert lassen, sonst kommen bald die Männer in weißen Kitteln
%
%bzw. für Freunde der kombinatorischen Logik
%\begin{displaymath}
%S S K (S (K (S S (S (S S K)))) K).
%\end{displaymath}

Aber zurück zum Thema Brot: Brot ist laut \begin{verbatim}http://de.wikipedia.org/w/index.php?title=Brot&oldid=95442475\end{verbatim} ein traditionelles Nahrungsmittel, das aus einem Teig aus gemahlenem Getreide (Mehl), Wasser, einem Triebmittel und meist weiteren Zutaten gebacken wird.

\end{document}
